\chapter{Estrutura}
	Este capítulo descreve alguns dos mecanismos de baixo nível que os provedores de nuvem utilizam para prover seus serviços. No modelo de IaaS por exemplo a nuvem provisiona ao cliente uma máquina virtual carregada com uma imagem. O conceito de virtualização é antigo porém é extremamente importante no contexto de cloud computing. Nas secções seguintes descreveremos os mecanismos principais.


	\section{Hypervisor}
	Um \textit{hypervisor} é uma espécie de sistema operacional usado para criar e gerenciar máquinas virtuais. Como cada máquina tem seu próprio sistema operacional, uma aplicação de um cliente é gerenciada por duas camadas de sistema operacional: o \textit{hypervisor} e o sistema operacional da máquina virtual. O \textit{hypervisor} administra o SO da máquina virtual e o SO da máquina virtual gerencia a aplicação do cliente. 

	Os principais serviços usados pelo \textit{hypervisor} para suportar o gerenciamento de máquinas virtuais são o mapeador de páginas virtuais (\textit{virtual page mapper}) e um agendador (\textit{scheduler}). Um \textit{hypervisor} também possui outros serviços, mas não o discutiremos neste trabalho.

	\subsection{Mapeador de páginas virtuais}	
	Usar memória virtual como um mecanismo de virtualização envolve adicionar outro nível de indireção. Os processadores modernos contêm muitas otimizações para tornar esse processo mais
	eficiente. 

	Um aplicativo de usuário gera a próxima instrução com seu endereço de destino. Este endereço de destino está dentro da VM na qual o aplicativo está sendo executado. A tabela de páginas da máquina mapeia esse endereço de destino para um endereço dentro da máquina virtual com base no endereço de destino como no caso normal. 

	O que é novo neste processo para nuvem é que o endereço dentro do VM é convertido em um endereço físico por uma tabela de páginas do \textit{hypervisor} que gerencia as VMs atuais.

	\subsection{Agendador}
	O agendador do \textit{hypervisor} opera como qualquer agendador de sistema operacional. Sempre que o \textit{hypervisor} tiver o controle de execução ele decide para qual máquina virtual passará o controle. Um simples agendamento de pelo algoritmo de \textit{round robin} atribui o processador a cada VM, mas muitas outras opções de algoritmos são possíveis. 

	A escolha do algoritmo de agendamento correto requer que você faça suposições sobre as características de demanda das diferentes VMs hospedadas em um único servidor. Uma área de pesquisa é a aplicação de algoritmos de programação em tempo real para \textit{hypervisors}. Agendadores em tempo real é apropriado para o uso de virtualização dentro de sistemas embarcados, mas não necessariamente na nuvem ainda.


\iffalse
	
	\section{Hadoop Distributed File System}
	

\fi