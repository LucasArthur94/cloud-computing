\chapter{Estrutura}

\section{Hypervisor}
\section{Orquestrador}
\section{Armazenamento}
\section{Rede}

\chapter{Consideração técnicas sobre requisitos}
	Neste capítulo apresentamos algumas considerações técnicas e táticas para cumprir com êxito três principais requisitos não funcionais claros para Cloud Computing: disponibilidade, performance e segurança.

\section{Disponibilidade}

	Assume-se que a nuvem sempre estará disponível, 24 horas por dia, 7 dias por semana. Porém nenhum de seus componentes é a prova de falhas. O host físico de uma máquina virtual pode falhar, bem como a rede virtual com menor probabilidade, ao sofrer um ataque de negação de serviço (DDoS), por exemplo.

	Os provedores de serviço na nuvem fornecem dados aos consumidores sobre a porcentagem de tempo que a rede pode ficar disponível. Apesar de ser um número pequeno (para alguns planos de serviço da AWS variam entre 0,05\% e 0,001\%), é de responsabilidade do arquiteto do sistema que consumidor projetá-lo de tal forma que seja incluído um plano de contingência para essa indisponibilidade.

	Discutiremos algumas táticas a seguir:

	\subsection{Stateless}
	Arquitear sistemas \textit{stateless} (Sem estado) permite que qualquer instância de serviço possa servir igualmente uma requisição do cliente. Assim, se um servidor falha as requisições são redirecionadas para outra instância de serviço.

	\subsection{Instâncias de reserva}
	O sistema deve incluir instâncias reservas da aplicação, para onde o tráfego de rede será redirecionado caso as instâncias principais venham a falhar.

	\subsection{Dados em zonas distintas}
	Um provedor na nuvem pode disponibilizar zonas distintas, que correspondem a \textit{data centers} distintos. O sistema então pode implementar uma redundância com diversas cópias dos dados espalhados por essas zonas. Se a rede de uma zona cair, uma zona que está ociosa ou a mais próxima ativa assume seu lugar.
	
	\subsection{Degradação silenciosa}
 	Os princípios para lidar com falhas em aplicações se baseam em táticas de renovação ou remoção de serviços:
	
	\begin{itemize}
	\item
	\textbf{Restauração rápida}:
	Crie \textit{timeouts} curtos para testar periodicamente o status de cada componente, restabelencendo-o antes que o sistema sinta essa falha.

	\item
	\textbf{Downgrades}:
	Cada feature é projetada com capacidade de ser degradada ou reduzida para uma representação de menor qualidade. Ex.: \textit{streaming} de vídeos.

	\item
	\textbf{Remoção de features}:
	Se a rede estiver mais lenta que o normal e a \textit{feature} não for crítica, ela deve ser removida da página.

	\end{itemize}
	\iffalse
	%TODO: Escrever
	\subsection{Exemplo: Netflix's Simian Army}

	\subsection{Exemplo: Skype Reverse NAT}
	\fi

\iffalse
\section{Performance}
\section{Segurança}
\fi