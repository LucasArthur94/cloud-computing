\chapter{Modelos para Cloud Computing}

Este capítulo introduz brevemente os modelos de serviço de cloud computing que podem ser adotados por um provedor e os modelos de \textit{deployment}, nas secções seguintes.

\section{Modelos de serviços}

A teoria por trás dos serviços de computação em nuvem abrange três elementos principais: software, plataforma e infraestrutura. Temos os seguintes modelos de serviço:

\begin{itemize}
  \item \textbf{SaaS, Software as a Service} ("Software como um serviço") 

  Neste modelo é oferecido ao consumidor o uso de aplicações de um provedor que rodam sobre uma infraestrutura em nuvem. Estas aplicações são acessíveis a partir de vários dispositivos clientes por meio de uma interface simples, como um navegador da web, ou uma interface de por meio de um programa (mobile ou desktop). 

  O consumidor não gerencia ou controla a infraestrutura de nuvem por trás da aplicação, incluindo rede, servidores, sistemas operacionais, armazenamento ou capacidade da aplicação individual. São disponibilizadas apenas configurações do aplicativo específicas para aquele usuário individual.

  Alguns exemplos de SaaS são serviços de \textit{webmail}, \textit{streaming} de vídeos, conversão de arquivos e trabalho colaborativo com arquivos. Este modelo não será mais detalhado neste trabalho.

  \item \textbf{PaaS, Platform as a Service} ("Plataforma como um serviço") 

  A capacidade fornecida ao consumidor é de implantar na nuvem aplicações criadas por meio de linguagens de programação, bibliotecas, serviços e ferramentas suportadas pelo provedor. Tais aplicações podem ser criadas pelo próprio consumidor, ou consumidas por este.

  Assim como no modelo de SaaS, o consumidor geralmente não gerencia ou têm controle sobre a infraestrutura de nuvem por trás, incluindo rede, servidores, sistemas operacionais, ou armazenamento. Porém o consumidor as tem controle sobre os aplicativos implantados e possivelmente sobre definições de configuração para o ambiente de hospedagem do aplicativo.

  Como exemplos de provedores no mercado temos \textit{IBM Bluemix}, \textit{Heroku}, e \textit{Windows Azure Cloud}.

  \item \textbf{IaaS, Infrastructure as a Service} ("Infraestrutura como um serviço") 

  A capacidade oferecida ao consumidor é provisionar processamento, armazenamento, redes e outros recursos fundamentais de computação onde o consumidor é capaz de implantar e executar software arbitrário, incluindo-se sistemas e aplicações. O consumidor não gerencia nem controla a infraestrutura de nuvem por trás, mas tem controle sobre sistemas operacionais, armazenamento e aplicativos implantados. Possivelmente possui também um controle limitado de componentes de rede (por exemplo, firewalls de host). Geralmente acompanha serviços de máquinas virtualizadas.

  Alguns exemplos de provedores no mercado são \textit{Amazon Web Services}, \textit{Microsoft Azure}, \textit{Google Cloud} e \textit{VMware Cloud on AWS}. 
\end{itemize}

\section{Modelos de implantação}
Os modelos de implantação (\textit{deployment models}) a seguir delimitam as formas possíveis de tarifação dos serviços, as possíveis localizações da infraestrutura física e o público de escopo. São eles:

\begin{itemize}
  \item \textbf{Nuvem privada}: A infraestrutura de nuvem é disponibilizada para uso exclusivo por uma única organização, compreendendo vários consumidores (\textit{business units}). Pode ser propriedade e gerenciada pela própria organização, um terceiro ou alguma combinação de ambos, e pode existir dentro ou fora das instalações da organização.
  \item \textbf{Nuvem comunitária}: A infraestrutura em nuvem é de uso exclusivo por uma comunidade de consumidores de organizações que compartilham mesmos interesses (por exemplo, missão, requisitos de segurança, políticas e considerações de conformidade). Pode ser de propriedade e administrada por uma ou mais organizações da comunidade, um terceiro, ou alguma combinação de ambos, e pode existir dentro ou fora das instalações.
  \item \textbf{Nuvem pública}: A infraestrutura de nuvem é para o uso aberto pelo público em geral. Pode ser propriedade ou gerenciada por uma organização comercial, acadêmica ou governamental, ou alguma combinação destes. Existe dentro das instalações do provedor de nuvem.
  \item \textbf{Nuvem híbrida}: A infraestrutura de nuvem é uma composição de duas ou mais infraestruturas de nuvens distintas (privadas, comunitárias ou públicas) que permanecem como entidades únicas, mas unidas por tecnologia padronizada ou proprietária permitindo portabilidade de dados e aplicações (por exemplo, \textit{cloud bursting} para balanceamento de carga entre nuvens).
\end{itemize}
