\chapter{Estrutura}

\section{Hypervisor}
\section{Orquestrador}
\section{Armazenamento}
\section{Rede}

\chapter{Consideração técnicas sobre requisitos}
	Neste capítulo apresentamos algumas considerações técnicas para cumprir com êxito três principais requisitos não funcionais claros para Cloud Computing: disponibilidade, performance e segurança.

\section{Disponibilidade}

	Assume-se que a nuvem sempre estará disponível, 24 horas por dia, 7 dias por semana. Porém nada é a prova de falhas. 
The cloud is assumed to be always available. But everything can fail. A virtual machine, for example, is
hosted on a physical machine that can fail. The virtual network is less likely to fail, but it too is fallible.
It behooves the architect of a system to plan for failure.
The service-level agreement that Amazon provides for its EC2 cloud service provides a 99.95
percent guarantee of service. There are two ways of looking at that number: ( 1 ) That is a high number.
You as an architect do not need to worry about failure. (2) That number indicates that the service may
be unavailable for .05 percent of the time. You as an architect need to plan for that .05 percent.
Netflix is a company that streams videos to home television sets, and its reliability is an important
business asset. N etflix also hosts much of its operation on Amazon EC2. On April 2 1 , 20 1 1 , Amazon
EC2 suffered a four-day sporadic outage. Netflix customers, however, were unaware of any problem.
Some of the things that N etflix did to promote availability that served them well during that period
were reported in their tech blog. We discussed their Simian Army in Chapter 10. Some of the other
things they did were applications of availability tactics that we discussed in Chapter 5 .
• Stateless services. Netflix services are designed such that any service instance can serve any
request in a timely fashion, so if a server fails, requests can be routed to another service
instance. This is an application of the spare tactic, because the other service instance acts as a
spare.
• Data stored across zones. Amazon provides what they call "availability zones," which are
distinct data centers. Netflix ensured that there were multiple redundant hot copies of the data
spread across zones. Failures were retried in another zone, or a hot standby was invoked. This
is an example of the active redundancy tactic.
• Graceful degradation. The general principles for dealing with failure are applications of the
degradation or the retnoval from service tactic:
• Fail fast: Set aggressive timeouts such that failing components don't make the entire system
crawl to a halt.
• Fallbacks: Each feature is designed to degrade or fall back to a lower quality representation.
• Feature removal: If a feature is noncritical, then if it is slow it may be removed from any
g1ven page.

\iffalse
\section{Performance}
\section{Segurança}
\fi