\chapter{Conceitos Básicos}

Este capítulo introduz as definições e os conceitos básicos quando se trata de Cloud Computing.

\section{Definição de Cloud Computing}

Computação em Nuvem (do Inglês \textit{Cloud Computing}) é segundo a NIST (\textit{National Institute of Standards and Technology}) um modelo para prover um acesso de rede a um grupo compartilhado (\textit{shared pool}) de recursos computacionais (recursos como capacidade de rede, servidores, armazenamento, aplicações e serviços) de forma ubíqua, prática e sob demanda.

Tal acesso deve ser rapidamente provisionado e lançado com o mínimo de gerenciamento e interação com o provedor de serviços por parte da aplicação.

O modelo de computação em nuvem deve possuir algumas características básicas, que estão descritas na secção seguinte.

\section{Características básicas de Cloud Computing}

A NIST define 5 características essenciais do modelo de Cloud Computing:
\begin{itemize}
	\item Serviço sob demanda: Um consumidor pode provisionar unilateralmente capacidades computacionais, como tempo de servidor e armazenamento de rede, conforme for necessário, sem qualquer interação humana com o provedor de serviço;
	\item Agrupamento de recursos: Os recursos de computação do provedor são agrupados para atender a vários consumidores usando um modelo "multi inquilino", com diferentes recursos físicos e virtuais dinamicamente atribuídos e reatribuídos de acordo com a demanda do consumidor. Existe uma sensação de independência de localização pois o cliente geralmente não tem controle ou conhecimento sobre a localização exata dos recursos fornecidos, mas pode ser capaz de especificar a localização em um nível de abstração (por exemplo, país, estado ou datacenter). Exemplos de recursos incluem armazenamento, processamento, memória e largura de banda de rede.
	\item Amplo acesso à rede: Os recursos estão disponíveis na rede e são acessados por meio de mecanismos que promovam o uso por plataformas heterogêneas por clientes em diversos dispositivos (por exemplo, telefones celulares, tablets, laptops e estações de trabalho). Esta característica promove o conceito de computação ubíqua, isto é, em toda parte, onipresente.
	\item Elasticidade rápida: Os recursos podem ser provisionados e liberados elasticamente, em alguns casos automaticamente, proporcionando uma escalabilidade crescente ou descrescente conforme a demanda. Os recursos disponíveis normalmente aparentam ser ilimitados para o consumidor, podendo ser requisitidados em qualquer quantidade e a qualquer momento.
	\item Serviço mensurável: Os sistemas em nuvem controlam e otimizam automaticamente o uso de recursos, aproveitando-se de uma capacidade de medição em um nível de abstração apropriado ao tipo de serviço (por exemplo, armazenamento, processamento, largura de banda e contas de usuário ativas). O uso de recursos pode ser monitorado, controlado e reportado, gerando transparência tanto para o fornecedor e consumidor do serviço utilizado.
\end{itemize}
